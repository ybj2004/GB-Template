%!TEX program = xelatex
\documentclass{GB-template}

% use parameter
\ICS{这里输入 ICS 号}
\fenlei{这里输入分类号}
\biaozhunhao{在这里输入标准号}
\daitibiaozhunhao{在这里输入代替标准号}
\name{标准名称}
\enname{标准名称的英文译名}
\conformity{(与国际标准一致性的标识)}
\department{国家标准发布部门}
\fabudate{XXXX-XX-XX}
\shishidate{XXXX-XX-XX}

\begin{document}
\bibliographystyle{gbt7714-2005}
\frontmatter
\renewcommand{\thepage}{\Roman{page}}
% titlepage
\maketitlepage
\patchchapter
\tableofcontents\listoffigures\listoftables
\restorechapter
\begin{preamble}{前言}
    前言和序言有什么差别呢?前言和序言有什么差别呢?前言和序言有什么差别呢?前言和序言有什么差别呢?前言和序言有什么差别呢?
\end{preamble}
\begin{preamble}{序言}
    前言和序言有什么差别呢?前言和序言有什么差别呢?前言和序言有什么差别呢?前言和序言有什么差别呢?前言和序言有什么差别呢?
\end{preamble}
\mainmatter
\chapter{模板的基本使用}
「要求」中大多数细节都能通过 \LaTeX{} 比较舒服地实现,唯独「索引」\index{索引}部分是个老大难。好在经过一番实践,最终是将之搞定,不过也因此给模板的使用带来了一些问题——主要是需要进行一些安装和配置。

除掉模板文件内包含的东西,还需要额外下载安装名为 Python\index{Python} 的脚本语言解释器。你可以在\href{https://www.python.org/ftp/python/2.7.6/}{\fbox{这里}}找到适用于全部平台的 Python。比如适用于 32 位 Windows 系统的版本是\href{https://www.python.org/ftp/python/2.7.6/python-2.7.6.msi}{\fbox{这个}},而 64 位的 Windows 系统则应该安装\href{https://www.python.org/ftp/python/2.7.6/python-2.7.6.amd64.msi}{\fbox{这个}}。

此外,为了让 \LaTeX{} 能够载入外部程序,在执行 \hologo{XeLaTeX} 的时候,还必须给它加上 \verb|--shell-escape| \index{-{}-shell-escape}选项,所以整个命令看起来像是下面这样:

\begin{quote}
\begin{verbatim}
xelatex --shell-escape GB-template
bibtex GB-template
xelatex --shell-escape GB-template
xelatex --shell-escape GB-template
\end{verbatim}
\end{quote}

当然,这有点复杂了。所以我准备了名为 \verb|make.bat| 的批处理文件,自动做这些工作。你只需要在编写好 \LaTeX{} 文档之后,双击运行这个文件就会生成 PDF 了。

最后,为了保证文章内没有乱码,所有的文件必须使用 UTF-8 编码\index{编码}来保存。这一点尤为重要,但却不是我所能控制的了,因为这是「用户」使用时候应当注意的问题。

\chapter{章、条的}
\section{编号和标题}
\subsection{用五号黑体}
\subsubsection{标注条文、列项及其编号用五号宋体}
\begin{figure}[htb]
\centering
\rule{4cm}{3cm}
\caption{图表的标题用 5 号黑体字}\label{fig:dummyfigure}
\end{figure}

让我们看看脚注的样子\footnote{脚注的标号应该是小五号宋体。如果脚注特别长的话,换行应该对齐。如果脚注特别长的话,换行应该对齐。如果脚注特别长的话,换行应该对齐。如果脚注特别长的话,换行应该对齐。如果脚注特别长的话,换行应该对齐\index{对齐}。如果脚注特别长的话,换行应该对齐。如果脚注特别长的话,换行应该对齐。}。

哦,让我们再来说一说序言的故事。这其实是为了测试索引\index{索引}的效果。

\chapter{这里}
\section{用来}
\subsection{测试}
\subsubsection{的效果}
这里用来测试索引\index{索引}的效果\index{效果}。
\subsection{再来一个小结}
再来一个索引\index{索引}。
\appendix
\autoclearpage
\gchapter{规范型附录标题}
\zchapter{资料性附录标题}
\end{document}
